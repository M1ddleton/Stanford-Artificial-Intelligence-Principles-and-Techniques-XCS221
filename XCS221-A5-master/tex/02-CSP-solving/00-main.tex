\item {\bf CSP Solving}

So far, we've only worked with unweighted CSPs, where $f_j(x)\in\{0,1\}$. In
this problem, we will work with weighted CSPs, which associates a weight for
each assignment $x$ based on the product of $m$ factor functions $f_1, \dots,
f_m$:
\[\text{Weight}(x) = \prod^m_{j=1}f_j(x)\]
where each factor $f_j(x)\geq 0$. Our goal is to find the assignment(s) $x$ with
the {\bf highest} weight. As in problem 0, we will assume that each factor is
either a unary factor (depends on exactly one variable) or a binary factor
(depends on exactly two variables).

For weighted CSP construction, you can refer to the CSP examples we have
provided in |util.py| for guidance (|create_map_coloring_csp()| and
|create_weighted_csp()|). You can try these examples out by running

\begin{lstlisting}
python run_p1.py
\end{lstlisting}

Notice we are already able to solve the CSPs because in |submission.py|, a basic
backtracking search is already implemented. Recall that backtracking search
operates over partial assignments and associates each partial assignment with a
weight, which is the product of all the factors that depend only on the assigned
variables. When we assign a value to a new variable $X_i$, we multiply in all
the factors that depend only on $X_i$ and the previously assigned variables. The
function |get_delta_weight()| returns the contribution of these new factors
based on the |unaryFactors| and |binaryFactors|. An important case is when
|get_delta_weight()| returns 0. In this case, any full assignment that extends
the new partial assignment will also be zero, so {\em there is no need to search
further with that new partial assignment}.

Take a look at |BacktrackingSearch.reset_results()| to see the other fields
which are set as a result of solving the weighted CSP. You should read
|submission.BacktrackingSearch| carefully to make sure that you understand how
the backtracking search is working on the CSP.

\begin{enumerate}

  \item \points{1a}
Let's create a CSP to solve the n-queens problem: Given an $n\times n$ board,
we'd like to place $n$ queens on this board such that no two queens are on the
same row, column, or diagonal. Implement |create_nqueens_csp()| by {\bf adding
$n$ variables} and some number of binary factors. Note that the solver collects
some basic statistics on the performance of the algorithm. You should take
advantage of these statistics for debugging and analysis. You should get 92
(optimal) assignments for $n=8$ with exactly 2057 operations (number of calls to
|backtrack()|). 

{\em Hint: If you get a larger number of operations, make sure your CSP is
minimal. Try to define the variables such that the size of domain is O(n).

Note: Please implement the domain of variables as 'list' type in Python (again,
you may refer to |create_map_coloring_csp()| and |create_weighted_csp()| in
|util.py| as examples of CSP problem implementations), so you can compare the
number of operations with our suggestions as a way of debugging.}


  \item \points{1b}
You might notice that our search algorithm explores quite a large number of
states even for the $8\times 8$ board. Let's see if we can do better. One
heuristic we discussed in class is using most constrained variable (MCV): To
choose an unassigned variable, pick the $X_j$ that has the fewest number of
values $a$ which are consistent with the current partial assignment ($a$ for
which |get_delta_weight()| on $X_j=a$ returns a non-zero value).

Implement this heuristic in |get_unassigned_variable()| under the condition
|self.mcv = True|. It should take you exactly 1361 operations to find all
optimal assignments for 8 queens CSP --- that's 30\% fewer!

Some useful fields:
\begin{itemize}
  \item |csp.unaryFactors[var][val]| gives the unary factor value.
  \item |csp.binaryFactors[var1][var2][val1][val2]| gives the binary factor
  value. Here, |var1| and |var2| are variables and |val1| and |val2| are their
  corresponding values.
  \item In |BacktrackingSearch|, if |var| has been assigned a value, you can
  retrieve it using |assignment[var]|. Otherwise |var| is not in |assignment|.
\end{itemize}


\end{enumerate}
