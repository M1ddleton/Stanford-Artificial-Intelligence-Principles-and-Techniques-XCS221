\item {\bf Course Scheduling}

In this problem, we will apply your weighted CSP solver to the problem of course
scheduling. We have scraped a subset of courses that are offered from Stanford's
Bulletin. For each course in this dataset, we have information on which quarters
it is offered, the prerequisites (which may not be fully accurate due to
ambiguity in the listing), and the range of units allowed. You can take a look
at all the courses in |courses.json|. Please refer to |util.Course| and
|util.CourseBulletin| for more information.

To specify a desired course plan, you would need to provide a {\em profile}
which specifies your constraints and preferences for courses. A profile is
specified in a text file (see |profile*.txt| for examples). The profile file has
four sections:
\begin{itemize}
  \item The first section specifies a fixed minimum and maximum (inclusive)
  number of units you need to take for each quarter. For example:
\begin{lstlisting}
minUnits 0
maxUnits 3
\end{lstlisting}

  \item In the second section, you |register| for the quarters that you want to
  take your courses in.  For example,
\begin{lstlisting}
register Aut2018
register Win2019
register Spr2019
\end{lstlisting}

  would sign you up for this academic year. The quarters need not be contiguous,
  but they must follow the exact format |XxxYYYY| where |Xxx| is one of
  |Aut, Win, Spr, Sum| and |YYYY| is the year.
  
  \item The third section specifies the list of courses that you've taken in the
  past and elsewhere using the |taken| keyword. For example, if you're in CS221,
  this is probably what you would put:
\begin{lstlisting}
taken CS103
taken CS106B
taken CS107
taken CS109
\end{lstlisting}

  \item The last section is a list of courses that you would like to take during
  the registered quarters, specified using |request|. For example, two basic
  requests would look like this:
\begin{lstlisting}
request CS224N
request CS229
\end{lstlisting}

Not every request must be fulfilled, and indeed, due to the additional
constraints described below, it is possible that not all of them can actually be
fulfilled.
\end{itemize}

{\bf Constrained requests.}
To allow for more flexibility in your preferences, we allow some freedom to
customize the requests:
\begin{itemize}
  \item You can request to take exclusively one of several courses but not sure
  which one, then specify:
\begin{lstlisting}
request CS229 or CS229A or CS229T
\end{lstlisting}


  Note that these courses do not necessarily have to be offered in the same
  quarter. The final schedule can have at most one of these three courses.
  {\bf Each course can only be requested at most once.}

  \item If you want to take a course in one of a specified set of quarters, use
  the |in| modifier. For example, if you want to take one of CS221 or CS229 in
  either Aut2018 {\bf or} Sum2019, do:
\begin{lstlisting}
request CS221 or CS229 in Aut2018,Sum2019
\end{lstlisting}

  If you do not specify any quarters, then the course can be taken in any quarter.

  \item Another operator you can apply is |after|, which specifies that a course
  must be taken after another one. For example, if you want to choose one of
  CS221 or CS229 and take it after both CS109 {\bf and} CS161, add:
\begin{lstlisting}
request CS221 or CS229 after CS109,CS161
\end{lstlisting}

  Note that this implies that if you take CS221 or CS229, then you must take
  both CS109 and CS161. In this case, we say that CS109 and CS161 are |prereqs|
  of this request. (Note that there's {\bf no space} after the comma.)

  If you request course A and B (separately), and A is an official prerequisite
  of B based on the |CourseBulletin|, we will automatically add A as a
  prerequisite for B; that is, typing |request B| is equivalent to
  |request B after A|. Note that if B is a prerequisite of A, to request A, you
  must either request B or declare you've taken B before.

  \item Finally, the last operator you can add is |weight|, which adds
  non-negative weight to each request. All requests have a default weight
  value of 1. Requests with higher weight should be preferred by your CSP
  solver. Note that you can combine all of the aforementioned operators into one
  as follows (again, no space after comma):
\begin{lstlisting}
request CS221 or CS229 in Win2018,Win2019 after CS131 weight 5
\end{lstlisting}
\end{itemize}

Each |request| line in your profile is represented in code as an instance of the
|Request| class (see |util.py|). For example, the request above would have the
following fields:
\begin{itemize}
    \item |cids| (course IDs that you're choosing one of) with value
    |['CS221', 'CS229']|
    \item |quarters| (that you're allowed to take the courses) with value |['Win2018', 'Win2019']|
    \item |prereqs| (course IDs that you must take before) with value |['CS131']|
    \item |weight| (preference) with value |5.0|
\end{itemize}

It's important to note that a request does not have to be fulfilled,
{\em but if it is}, the constraints specified by the various operators
|after,in| must also be satisfied.

You shall not worry about parsing the profiles because we have done all the
parsing of the bulletin and profile for you, so all you need to work with is the
collection of |Request| objects in |Profile| and |CourseBulletin| to know when
courses are offered and the number of units of courses.

Well, that's a lot of information! Let's open a python shell and see them in
action:
\begin{lstlisting}
(XCS221) $ python
Python 3.6.9
Type "help", "copyright", "credits" or "license" for more information.
>>> import util
>>> # load bulletin
...
>>> bulletin = util.CourseBulletin(`courses.json')
>>> # retrieve information of CS221
...
>>> cs221 = bulletin.courses[`CS221']
>>> print(cs221)
(look at various properties of the course)
>>> print(cs221.cid)
CS221
>>> print(cs221.minUnits)
3
>>> print(cs221.maxUnits)
4
>>> print(cs221.prereqs)  # the prerequisites
[`CS107', `CS103', `CS106X', `CS106B']
>>> print(cs221.is_offered_in(`Aut2018'))
True
>>> print(cs221.is_offered_in(`Win2019'))
False
>>> # load profile from profile_example.txt
...
>>> profile = util.Profile(bulletin, `profile_example.txt')
>>> # see what it's about
...
>>> profile.print_info()
Units: 3-9
Quarter: [`Aut2017', `Win2018']
Taken: {'CS229'}
Requests:
  Request{[`CS228'] [`Aut2017'] [] 1}
  Request{[`CS229T'] [] [`CS228'] 2.0}
>>> # iterate over the requests and print out the properties
...
>>> for request in profile.requests:
...   print(request.cids, request.quarters, request.prereqs, request.weight)
...
[`CS228'] [`Aut2017'] [] 1
[`CS229T'] [] [`CS228'] 2.0
>>> exit()
\end{lstlisting}

{\bf Solving the CSP.}

Your task is to take a profile and bulletin and construct a CSP. We have started
you off with code in |SchedulingCSPConstructor| that constructs the core
variables of the CSP as well as some basic constraints. The variables are all
pairs of requests and registered quarters |(request, quarter)|, and the value of
such a variable is one of the course IDs in that Request or |None|, which
indicates none of the courses should be taken in that quarter. We will add
auxiliary variables later. We have also implemented some basic constraints:
|add_bulletin_constraints()|, which enforces that a course can only be taken if
it's offered in that quarter (according to the bulletin), and
|add_norepeating_constraints()|, which constrains that no course can be taken
more than once.

You should take a look at |add_bulletin_constraints()| and 
|add_norepeating_constraints()| to get a basic understanding how the CSP for
scheduling is represented. Nevertheless, we'll highlight some important details
to make it easier for you to implement:



\begin{itemize}
  \item The existing variables are tuples of |(request, quarter)| where
  |request| is a |Request| object (like the one shown above) and |quarter| is a
  |str| representing a quarter (e.g. |'Aut2018'|). For detail please look at
  |SchedulingCSPConstructor.add_variables()|.
  
  \item The domain of each variable |(request, quarter)| is the course IDs of
  the request {\bf plus} |None| (e.g. |['CS221', 'CS229', None]|). When the
  value |cid| is |None|, this means no course is scheduled for this request.
  {\bf Always remember to check if |cid| is |None|}.
  
  \item The domain for |quarter| is all possible quarters
  (|self.profile.quarters|, e.g. |['Win2016', 'Win2017']|).
  
  \item Given a course ID |cid|, you can get the corresponding |Course| object
  by |self.bulletin.courses[cid]|.
\end{itemize}

\begin{enumerate}

  \item \points{2a}
Implement the function |add_quarter_constraints()| in |submission.py|. This is
when your profile specifies which quarter(s) you want your requested courses to
be taken in. This is not saying that one of the courses must be taken,
{\em but if it is}, then it must be taken in any one of the specified quarters.
Also note that this constraint will apply to all courses in that request.

We have written a |verify_schedule()| function in |grader.py| that determines if
your schedule satisfies all of the given constraints. Note that since we are not
dealing with units yet, it will print |None| for the number of units of each
course. For profile2a.txt, you should find 3 optimal assignments with weight 1.0.


  \item \points{2b}
Let's now add the unit constraints in |add_unit_constraints()|. (1) You must
ensure that the sum of units per quarter for your schedule are within the min
and max threshold inclusive. You should use |get_sum_variable()|. (2) In order
for our solution extractor to obtain the number of units, for every course, you
must add a variable |(courseId, quarter)| to the CSP taking on a value equal to
the number of units being taken for that course during that quarter. When the
course is not taken during that quarter, the unit should be 0.

{\em NOTE: Each grader test only tests the function you are asked to implement.
To test your CSP with multiple constraints you can use |run_p2.py| and add
whichever constraints that you want to add. For profile2b.txt, you should find
15 optimal assignments with weight 1.0.

Hint: If your code times out, your |maxSum| passed to |get_sum_variable()| might
be too large.}


  \input{03-course-scheduling/03-profile}

\end{enumerate}
